%! TEX root = ichthyosauromorphtaxonomy.tex

\section{Introduction}

This document presents a list of valid ichthyosauromorph species. It is a
constant work in progress given the pace at which new ichthyosaur research
happens. Perhaps not as rapidly as some other fossil groups, but certainly fast
enough to require some attention.

It's also intended to give some handle on the vast literature available for
ichthyosaurs and their near relatives by means of synonymy lists. I've chosen
this format for it's familiarity, but also because I rather enjoy the process
and trials to building a complete yet usable synonymy list.

I've used this also as an exercise in typography – like much of my more public
output. This document is typeset in \LuaLaTeX{} using typefaces with various
nice features – old style figures, small capitals, optical sizes. The fonts used
are nonetheless open source, courtesy of their development at Adobe.

\subsection{Layout of synonymy lists}%
\label{ssec:key-synonymy-lists}

The synonymy lists are presented as unruled tables:

\begin{examplebox}%[enhanced, frame hidden, borderline west = {2pt}{2pt}{red}, colback = white!00, width = 666pt] 
    \emph{Cartorhynchus lenticarpus}~\cauthyr{Motani2015N} \\
    {\footnotesize\hspace{5em}\href{http://zoobank.org/urn:lsid:zoobank.org:act:FCCC9BB7-FD52-42F4-B2EC-B0B7E2A1CA32}{\textallsc{urn:lsid:zoobank.org:act:FCCC9BB7-FD52-42F4-B2EC-B0B7E2A1CA32}}}\vspace{0.5\baselineskip}
    
    {\footnotesize
    \begin{tabular}{p{4em}p{4em}p{20em}p{42em}}
    <status> & <year> & \emph{<Taxon>} <Authority, Year> & <reference, page> [<occurrence information>.] <comments> \\
    *        & \cite*{Motani2015N} & \emph{Cartorhynchus lenticarpus} \cite{Motani2015N} \lsid{urn:lsid:zoobank.org:act:FCCC9BB7-FD52-42F4-B2EC-B0B7E2A1CA32} & \crefauth{Motani2015N} p~485 [Upper Member, Nanlinghu Formation, \emph{Subcolumbites} Ammonite Biozone (Olenekian, Lower Triassic, Triassic); Majishan Quarry, Chaohu City, Hefei, China (\textallsc{UTM WGS84 50R 577953 3499041 = 31° 37′ 26″ N 117° 49′ 19″ E}).] \textallsc{LSID:} \href{http://zoobank.org/urn:lsid:zoobank.org:pub:9CFFEE63-2B8A-4C01-B9C7-CD3C53D684F5}{\textallsc{urn:lsid:zoobank.org:pub:9CFFEE63-2B8A-4C01-B9C7-CD3C53D684F5}} \\
    \end{tabular}
    }
\end{examplebox}

If a work names a new species or combination, the \textallsc{LSID} of the act is
included as a link button alongside the taxon
name,\sidenote{i.e.~\lsid{urn:lsid:zoobank.org:act:FCCC9BB7-FD52-42F4-B2EC-B0B7E2A1CA32}}
where this is available. Similarly all works that have publication
\textallsc{LSID}s have those included too.


\subsection{‘Richter symbols’ \& certainty of assignment}%
\label{sub:richter-symbols}

I've followed the recommendations of \textcite{Matthews1973P}, pretty much to the
letter, for the symbols and styles included in the leftmost two columns.

\begin{description}
    \item[In front of the year] \par
    \begin{description}
        \item[*] (asterisk) this publication marks the name becoming valid under
            \textallsc{ICZN} rules.
        \item[.] (dot/period) we accept responsibility for attaching this reference
            to the taxon in question.
        \item[] (no sign) we cannot responsibly attach this reference, but do not
            doubt it.
        \item[?] (question mark) there is some doubt in attaching this reference to
            the current taxon.
        \item[v] \emph{vidimus,} we have checked the deposited specimens. Can be
            accompanied by the above tokens:
            \begin{description}
                \item[v*] we have seen the type specimen(s).
                \item[v.] we take responsibility for attaching the deposited specimens.
                \item[v] we do not take responsibility for attaching the deposited
                    specimens.
                \item[v?] the specimens cannot be certainly assigned to this taxon.
            \end{description}
        \item[(?)] the year of publication is uncertain.
        \item[p] only part of the deposited specimens can be assigned to the current
            taxon.
            \begin{description}
                \item[vp] the deposited specimens have been checked and only part of
                    them belong to this taxon.
            \end{description}
    \end{description}

\item[And by formatting the year] \par
    \begin{description}
        \item[1881] (italicised) this work does not add morphological
            information, only occurrence information.
        \item[\emph{1881}] (upright) the work adds to our knowledge on this taxon.
        \item[\emph{(1881)}] (parentheses surrounding the year) the date of this work is
            uncertain.
    \end{description}
\end{description}

\subsection{Life Science Identifiers \textallsc{(LSID)}}%
\label{ssec:introduction-lsid}

\emph{Life science
identifiers}\sidenote{\url{https://en.wikipedia.org/wiki/LSID}} are unique keys
to identify and locate information important to the various life sciences.
Relevant to this document are the identifiers used to link nomenclatural acts in
\emph{ZooBank},\sidenote{\url{http://zoobank.org}} the official registry of the International
Commission on Zoological Nomenclature \textallsc{(ICZN).} Naming new taxa in the
recent literature requires registering the act in ZooBank to be ‘officially’
recognised.\sidenote{\url{https://www.iczn.org/the-code/the-international-code-of-zoological-nomenclature/the-code-online/}
(Article 8.5.3);\url{https://en.wikipedia.org/wiki/ZooBank}}

Many of the taxa included herein were named before ZooBank was yet a glint in
anyone's eye, although they may well have been included in the printed
equivalent \emph{Zoological Record.} I've registered several new publications
and acts where these were not in ZooBank before, that may be used henceforth. At
the moment, these include only the original naming of species, not new
combinations of specific and generic names.

\subsection{Links in this document}%
\label{sub:introduction-links}

The \LaTeX{} document from which this \textallsc{PDF} is compiled includes the
\emph{hyperref} package to provide links within the
document\sidenote{e.g.~citations and sections} and to other references on the
web.\sidenote{e.g.~\textallsc{LSID}s and \textallsc{DOI}s} These have different
colours, as follows:

\begin{description}
    \item[\textcolor{citelinkcolour}{Citation}] links from the year of
        a citation to its reference in the bibliography.
    \item[\textallsc{\textcolor{urllinkcolour}{URL}}] often web links,
        particularly to websites, \textallsc{DOI} links, \textallsc{LSID} with
        the address fully written.
    \item[{\upshape\lsid{x}}] link to a ZooBank \textallsc{LSID} for an act.
\end{description}

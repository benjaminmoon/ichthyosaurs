% !TEX program = lualatex
% !BIB program = biber
% !TEX spellcheck = en_GB
% !TEX encoding = utf8

\documentclass[notuftebib,font=source,UKenglish]{tufte-lualatex}

\setdefaultlanguage[variant=british]{english}


\setlength\LTpre{0.5\baselineskip}
\ProvideDocumentEnvironment{synonymy}
    {}%
    {
        \footnotesize
        \begin{longtable}[l]{@{\hspace{0em}}p{1.5em}@{\hspace{0.5em}}p{2.5em}@{\hspace{0.5em}}p{22em}@{\hspace{0.5em}}p{42em}}
    }%
    {
        \end{longtable}
    }

\usepackage{multicol}

\usepackage{parskip}

% \usepackage{tikz}
\usepackage{tcolorbox}
\tcbuselibrary{skins}
\definecolor{lsid_link}{Hsb}{25, 0.8, 0.9}
\colorlet{lsid_link}[rgb]{lsid_link}

\newtcbox{\lsidbox}{on line, colback = lsid_link, arc = 0.2em, left = -0.15em,
    right = -0.15em, top = -0.15em, bottom = -0.15em, boxrule = 0.0pt}

\ProvideDocumentCommand{\lsid}{ m }{
    \lsidbox{{\sffamily\bfseries\footnotesize\href{http://zoobank.org/#1}{\textcolor{white}{\textallsc{LSID}}}}}
}

% \usepackage{tocloft}
\NewDocumentCommand{\listspeciesname}{}{List of Species}
% \newlistof{species}{losp}{\listspeciesname}
\NewDocumentCommand{\species}{ m }{%
    % \refstepcounter{species}%
    \emph{#1}\markboth{#1}{#1}%
    \addcontentsline{losp}{species}{#1}%
}
% \cftsetindents{species}{0em}{0em}
\contentsuse{species}{losp}
\makeatletter
\titlecontents{species}
    [0em]
    {}
    {\footnotesize\itshape}
    {\footnotesize\itshape}
    {\footnotesize\hspace{1em}\tuftelltx@accenttext{\contentspage}}
\makeatother

% \ProvideDocumentCommand{\species}{ m }{\emph{#1}\markboth{#1}{#1}}

%% bibliography
\usepackage[
    backend      = biber,
	sortlocale   = auto,
	style        = authoryear,
	hyperref     = auto,
	maxcitenames = 2,
	mincitenames = 1,
	maxbibnames  = 100,
	alldates     = year,
	uniquename   = false,
	uniquelist   = false,
	backref      = false,
	giveninits   = true,
	dashed       = false,
	mergedate    = maximum,
	doi          = true,
	sorting      = nyt%
]{biblatex}
\addbibresource{synonymy.bib}

\RenewDocumentCommand{\bibfont}{}{\small\sffamily}
\RenewDocumentCommand{\UrlFont}{}{\small\ttfamily}

\ProvideDocumentCommand{\mywidebib}{}{
	\newgeometry{left = 3pc, textwidth = 44pc, top = 4pc, textheight = 60pc, nomarginpar}
	\begin{multicols}{3}[\printbibheading]
		\newrefcontext[sorting=nyt]
		\printbibliography[heading=none]
	\end{multicols}
	\restoregeometry
}

\RenewDocumentCommand{\nameyeardelim}{}{{, }}

\ProvideDocumentCommand{\cauth}{ m }{\cite{#1}}
\ProvideDocumentCommand{\crefauth}{ m }{\citeauthor{#1}}
\ProvideDocumentCommand{\pauth}{ m }{\parencite{#1}}
\ProvideDocumentCommand{\cyear}{ m }{\cite*{#1}}
\ProvideDocumentCommand{\pyear}{ m }{\parencite*{#1}}
\ProvideDocumentCommand{\emyear}{ m }{\emph{\cite*{#1}}}
\ProvideDocumentCommand{\cauthyr}{ m }{\cite{#1}}
\ProvideDocumentCommand{\pauthyr}{ m }{\parencite{#1}}

% show first and last species on page in footer – like an index
\AtBeginDocument{
    \fancyfoot[L]{\nouppercase{{\small\itshape\rightmark–\leftmark}}}
}
% single section in references
\AtBeginBibliography{
    \fancyfoot[L]{\nouppercase{{\itshape\leftmark}}}
}

\setVersion{0.5} %version number

\title{Ichthyosauromorph taxonomy}
\author{Benjamin Moon}
\date{\sffamily Version~\version.\isodash{}\isodate\today}

\begin{document}

\maketitle

\tableofcontents

\section{Introduction}

This document presents a list of valid ichthyosauromorph species. It is a
constant work in progress given the pace at which new ichthyosaur research
happens. Perhaps not as rapidly as some other fossil groups, but certainly fast
enough to require some attention.

It's also intended to give some handle on the vast literature available for
ichthyosaurs and their near relatives by means of synonymy lists. I've chosen
this format for it's familiarity, but also because I rather enjoy the process
and trials to building a complete yet usable synonymy list.

I've used this also as an exercise in typography – like much of my more public
output. This document is typeset in \LuaLaTeX{} using typefaces with various
nice features – old style figures, small capitals, optical sizes. The fonts used
are nonetheless open source, courtesy of their development at Adobe.

\subsection{Layout of synonymy lists}%
\label{ssec:key-synonymy-lists}

The synonymy lists are presented as unruled tables:

\begin{tcolorbox}[enhanced, frame hidden, borderline west = {2pt}{2pt}{red}, colback = white!00, width = 666pt] 
    \emph{Cartorhynchus lenticarpus}~\cauthyr{Motani2015N} \\
    {\footnotesize\hspace{5em}\textallsc{LSID:} \href{http://zoobank.org/urn:lsid:zoobank.org:act:FCCC9BB7-FD52-42F4-B2EC-B0B7E2A1CA32}{\textallsc{urn:lsid:zoobank.org:act:FCCC9BB7-FD52-42F4-B2EC-B0B7E2A1CA32}}}\vspace{0.5\baselineskip}
    
    {\footnotesize
    \begin{tabular}{p{4em}p{4em}p{20em}p{42em}}
    <status> & <year> & \emph{<Taxon>} <Authority, Year> & <reference, page> [<occurrence information>.] <comments> \\
    *        & \cite*{Motani2015N} & \emph{Cartorhynchus lenticarpus} \cite{Motani2015N} \lsid{urn:lsid:zoobank.org:act:FCCC9BB7-FD52-42F4-B2EC-B0B7E2A1CA32} & \crefauth{Motani2015N} p~485 [Upper Member, Nanlinghu Formation, \emph{Subcolumbites} Ammonite Biozone (Olenekian, Lower Triassic, Triassic); Majishan Quarry, Chaohu City, Hefei, China (\textallsc{UTM WGS84 50R 577953 3499041 = 31° 37′ 26″ N 117° 49′ 19″ E}).] \textallsc{LSID:} \href{http://zoobank.org/urn:lsid:zoobank.org:pub:9CFFEE63-2B8A-4C01-B9C7-CD3C53D684F5}{\textallsc{urn:lsid:zoobank.org:pub:9CFFEE63-2B8A-4C01-B9C7-CD3C53D684F5}} \\
    \end{tabular}
    }
\end{tcolorbox}

If a work names a new species or combination, the \textallsc{LSID} of the act is
included as a link button alongside the taxon
name,\sidenote{i.e.~\lsid{urn:lsid:zoobank.org:act:FCCC9BB7-FD52-42F4-B2EC-B0B7E2A1CA32}}
where this is available. Similarly all works that have publication
\textallsc{LSID}s have those included too.


\subsection{‘Richter symbols’ \& certainty of assignment}%
\label{sub:richter-symbols}

I've followed the recommendations of \textcite{Matthews1973P}, pretty much to the
letter, for the symbols and styles included in the leftmost two columns.

\begin{description}
    \item[In front of the year] \par
    \begin{description}
        \item[*] (asterisk) this publication marks the name becoming valid under
            \textallsc{ICZN} rules.
        \item[.] (dot/period) we accept responsibility for attaching this reference
            to the taxon in question.
        \item[] (no sign) we cannot responsibly attach this reference, but do not
            doubt it.
        \item[?] (question mark) there is some doubt in attaching this reference to
            the current taxon.
        \item[v] \emph{vidimus,} we have checked the deposited specimens. Can be
            accompanied by the above tokens:
            \begin{description}
                \item[v*] we have seen the type specimen(s).
                \item[v.] we take responsibility for attaching the deposited specimens.
                \item[v] we do not take responsibility for attaching the deposited
                    specimens.
                \item[v?] the specimens cannot be certainly assigned to this taxon.
            \end{description}
        \item[(?)] the year of publication is uncertain.
        \item[p] only part of the deposited specimens can be assigned to the current
            taxon.
            \begin{description}
                \item[vp] the deposited specimens have been checked and only part of
                    them belong to this taxon.
            \end{description}
    \end{description}

\item[And by formatting the year] \par
    \begin{description}
        \item[1881] (italicised) this work does not add morphological
            information, only occurrence information.
        \item[\emph{1881}] (upright) the work adds to our knowledge on this taxon.
        \item[\emph{(1881)}] (parentheses surrounding the year) the date of this work is
            uncertain.
    \end{description}
\end{description}

\subsection{Life Science Identifiers \textallsc{(LSID)}}%
\label{ssec:introduction-lsid}

\emph{Life science
identifiers}\sidenote{\url{https://en.wikipedia.org/wiki/LSID}} are unique keys
to identify and locate information important to the various life sciences.
Relevant to this document are the identifiers used to link nomenclatural acts in
\emph{ZooBank},\sidenote{\url{http://zoobank.org}} the official registry of the International
Commission on Zoological Nomenclature \textallsc{(ICZN).} Naming new taxa in the
recent literature requires registering the act in ZooBank to be ‘officially’
recognised.\sidenote{\url{https://en.wikipedia.org/wiki/ZooBank}}

Many of the taxa included herein were named before ZooBank was yet a glint in
anyone's eye, although they may well have been included in the printed
equivalent \emph{Zoological Record.} I've registered several new publications
and acts where these were not in ZooBank before, that may be used henceforth. At
the moment, these include only the original naming of species, not new
combinations of specific and generic names.

\subsection{Links in this document}%
\label{sub:introduction-links}

The \LaTeX{} document from which this \textallsc{PDF} is compiled includes the
\emph{hyperref} package to provide links within the
document\sidenote{e.g.~citations and sections} and to other references on the
web\sidenote{e.g.~\textallsc{LSID}s and \textallsc{DOI}s}. These have different
colours, as follows:

\begin{description}
    \item[\textcolor{citelinkcolour}{Citation}] links from the year of
        a citation to its reference in the bibliography.
    \item[\textallsc{\textcolor{urllinkcolour}{URL}}] often web links,
        particularly to websites, \textallsc{DOI} links, \textallsc{LSID} with
        the address fully written.
    \item[{\upshape\lsid{x}}] link to a ZooBank \textallsc{LSID} for an act.
\end{description}

\begin{multicols}{3}
\listofspecies
\end{multicols}{3}

\clearpage


\textcaps{Ichthyosauromorpha} \cite*[Motani et al.,][]{Motani2015N}
\vspace{1em}

\normalsize
{\textsc{Hupehsuchia} \cite*[Carroll \& Dong,][]{Carroll1991PTRSBBS}}

\emph{Eohupehsuchus brevicollis} \cite*[Chen et al,][]{Chen2014POa}\\
\begin{synonymy}
	*	& \cite*{Chen2014POa}	& \emph{Eohupehsuchus brevicollis} Chen et al. & p~4 [Jialingjiang Formation (Upper Spathian, Lower Triassic); Yangping, Yuan’an County, Hubei Province, China].
\end{synonymy}

\emph{Eretmorhipis carrolldongi} \cite*[Chen et al.,][]{Chen2015PO}\\
\begin{synonymy}
	& \cite*{Carroll1991PTRSBBS}	& Hupehsuchia n.\ g., n.\ s. & p~143 [Jialingjiang Formation (Upper Spathian, Lower Triassic); Tuling, Baihechuan, Xunjian District, Nanzhang County, Hubei Province, China].\\
	{*}	& \cite*{Chen2015PO}	& \emph{Eretmorhipis carrolldongi} Chen et al. & p~4 [Jialingjiang Formation (Upper Spathian, Lower Triassic); Yingzhishang, Yuan’an County, Hubei Province, China].
\end{synonymy}

\emph{Hupehsuchus nanchangensis} \cite*[Young \& Dong,][]{Young1972MNIGP}

\emph{Nanchangosaurus suni} \cite*[Wang,][]{Wang1959APS}

\emph{Parahupehsuchus longus} \cite*[Chen et al.,][]{Chen2014PO}

\vspace{1em}


\textsc{Ichthyosauriformes} \cite*[Motani et al.,][]{Motani2015N}

%! TEX root = ichthyosauromorphtaxonomy.tex

\emph{Acamptonectes densus}~\cauthyr{Fischer2012PO}

\begin{synonymy}
v * & \cyear{Fischer2012PO} & \emph{Acamptonectes densus} \cauth{Fischer2012PO}& \crefauth{Fischer2012PO}, p~3 [Speeton Clay Formation, \emph{Simbiskites concinnus/staffi} ammonite biozones (basal–upper Hauterivian, Lower Cretaceous, Cretaceous); Speeton, Yorkshire, U.K. and Cremlingen, Lower Saxony, Germany.]  \\
\end{synonymy}

\emph{Aegirosaurus leptospondylus}~\pauthyr{Wagner1853BKAWGA}

\begin{synonymy}
* & \cyear{Wagner1853BKAWGA} & \emph{Ichthyosaurus leptospondylus} \cauth{Wagner1853BKAWGA}& \crefauth{Wagner1853BKAWGA}, p~ [Solnhofen Formation (Lower Tithonian, Upper Jurassic, Jurassic); Germany.]  \\
 & \cyear{Bardet2000JP} & \emph{Aegirosaurus leptospondylus} \pauth{Wagner1853BKAWGA}& \crefauth{Bardet2000JP}, p~504 [Malm ζ 2b, Solnhofen Formation (early Lower Tithonian, Upper Jurassic, Jurassic); Borscheim, Schrandel quarry district; Bavaria, Germany.]  \\
\end{synonymy}

\emph{Arthropterygius chrisorum}~\pauthyr{Russell1993BGSC}

\begin{synonymy}
* & \cyear{Russell1993BGSC} & \emph{Ophthalmosaurus chrisorum} \cauth{Russell1993BGSC}& \crefauth{Russell1993BGSC}, p~198 [Ringnes Formation (Oxfordian–Kimmeridgian, Upper Jurassic, Jurassic); Cape Grassy, Melville Island, Canada (\textallsc{UTM WGS84 12X 433220 8453461 = 76° 9′ 4″ N 113° 30′ W}).]  \\
? * & \cyear{Delsett2017PO} & \emph{Keilhauia nui} \cauth{Delsett2017PO}& \crefauth{Delsett2017PO}, p~7 [Slottsmøya Member, Agardfjellet Formation (early Berriasian, Lower Cretaceous, Cretaceous); Janusfjellet, Spitsbergen, Svalbard (\textallsc{UTM WGS84 33X 0518847 8696044}).] \emph{fide} \textcite{Zverkov2019P} \\
p & \cyear{Delsett2018P} & \emph{Palvennia hoybergeti} \cauth{Druckenmiller2012NJG}& \crefauth{Delsett2018P}, p~8 [\textallsc{UTM WGS84 33X 0519622 8695649}] \textallsc{PMO 222.669} \emph{fide} \textcite{Zverkov2019P} \\
 & \cyear{Zverkov2019P} & \emph{Arthropterygius chrisorum} \pauth{Russell1993BGSC}& \crefauth{Zverkov2019P}, p~15  \\
\end{synonymy}

\emph{Arthropterygius hoybergeti}~\pauthyr{Druckenmiller2012NJG}

\begin{synonymy}
* & \cyear{Druckenmiller2012NJG} & \emph{Palvennia hoybergeti} \cauth{Druckenmiller2012NJG}& \crefauth{Druckenmiller2012NJG}, p~326 [Slottsmøya Member, Agardfjellet Formation (Middle Volgian, Upper Jurassic, Jurassic); Spitsbergen, Svalbard (\textallsc{UTM WGS84 33N 0518775 8696150}).]  \\
p & \cyear{Delsett2018P} & \emph{Palvennia hoybergeti} \cauth{Druckenmiller2012NJG}& \crefauth{Delsett2018P}, p~8 [\textallsc{UTM WGS84 33X 0519622 8695649}]  \\
 & \cyear{Zverkov2019P} & \emph{Arthropterygius hoybergeti} \pauth{Druckenmiller2012NJG}& \crefauth{Zverkov2019P}, p~31  \\
\end{synonymy}

\emph{Arthropterygius lundi}~\pauthyr{Roberts2014PO}

\begin{synonymy}
* & \cyear{Roberts2014PO} & \emph{Janusaurus lundi} \cauth{Roberts2014PO}& \crefauth{Roberts2014PO}, p~4 [Slottsmøya Member, Agardfjellet Formation (middle Volgian, Upper Jurassic, Jurassic); Janusfjellet, Spitsbergen, Svalbard (\textallsc{UTM WGS84 33X 518821 8696195 = 78° 20.264′ N 15° 50.044′ E}).] \emph{fide} \textcite{Zverkov2019P} \\
 & \emyear{Delsett2016GSLSP} & \emph{Janusaurus lundi} \cauth{Roberts2014PO}& \crefauth{Delsett2016GSLSP}, p~  \\
 & \emyear{Delsett2017PO} & \emph{Janusaurus lundi} \cauth{Roberts2014PO}& \crefauth{Delsett2017PO}, p~  \\
 & \cyear{Zverkov2019P} & \emph{Arthropterygius lundi} \pauth{Roberts2014PO}& \crefauth{Zverkov2019P}, p~40  \\
\end{synonymy}

\emph{Athabascasaurus bitumineus}~\cauthyr{Druckenmiller2010CJES}

\begin{synonymy}
v * & \cyear{Druckenmiller2010CJES} & \emph{Athabascasaurus bitumineus} \cauth{Druckenmiller2010CJES}& \crefauth{Druckenmiller2010CJES}, p~1039 [Wabiskaw Member, Clearwater Formation (lowermost Albian, Lower Cretaceous, Cretaceous); ~35 km north of Fort McMurray, Alberta, Canada (\textallsc{UTM WGS84 12V 459464 6317120 = 56° 59′ 45″ N 111° 40′ 02″ W}).]  \\
\end{synonymy}

\emph{Barracudasauroides panxianensis}~\pauthyr{Jiang2006JVP}

\emph{Besanosaurus leptorhynchus}~\cauthyr{DalSasso1996PL}

\emph{Brachypterygius alekseevi}~\pauthyr{Arkhangelsky2001PJ}

\emph{Brachypterygius extremus}~\pauthyr{Boulenger1904PZSL}

\begin{synonymy}
v * & \cyear{Boulenger1904PZSL} & \emph{Ichthyosaurus extremus} \cauth{Boulenger1904PZSL}& \crefauth{Boulenger1904PZSL}, p~425 [Kimmeridge Clay Formation (Kimmeridgian–Tithonia, Upper Jurassic, Jurassic); Smallmouth Sands, Weymouth, Dorset, U.K..]  \\
v & \cyear{vonHuene1922} & \emph{Brachypterygius extremus} \pauth{Boulenger1904PZSL}& \crefauth{vonHuene1922}, p~97  \\
v * & \cyear{McGowan1976CJES} & \emph{Grendelius mordax} \cauth{McGowan1976CJES}& \crefauth{McGowan1976CJES}, p~671 [Kimmeridge Clay Formation, \emph{Aulacostephanus autissiodorensis} Ammonite Biozone (middle Kimmeridgian, Upper Jurassic, Jurassic); Stowbridge, Norfolk, U.K. (\textallsc{UTM WGS84 31U 321960 5835045}).]  \\
\end{synonymy}

\emph{Brachypterygius pseudoscythica}~\pauthyr{Efimov1998PZ}

\emph{Californosaurus perrini}~\pauthyr{Merriam1902UCBDG}

\emph{Callawayia neoscapularis}~\pauthyr{McGowan1994JVPa}

\emph{Cartorhynchus lenticarpus}~\cauthyr{Motani2015N}

\emph{Caypullisaurus bonapartei}~\cauthyr{Fernandez1997JP}

\emph{Cetarthrosaurus walkeri}~\cauthyr{Seeley1873QJGS}

\emph{Chacaicosaurus cayi}~\cauthyr{Fernandez1994A}

\emph{Chaohusaurus chaoxianensis}~\pauthyr{Chen1985RGC}

\emph{Chaohusaurus geishanensis}~\cauthyr{Young1972MNIGP}

\emph{Chaohusaurus zhangjiawanensis}~\cauthyr{Chen2013AGS}

\emph{Contectopalatus atavus}~\pauthyr{Quenstedt1852}

\emph{Cymbospondylus buchseri}~\cauthyr{Sander1989JVP}

\emph{Cymbospondylus nichollsi}~\cauthyr{Frobisch2006ZJLS}

\emph{Cymbospondylus petrinus}~\cauthyr{Leidy1868PANSP}

\emph{Cymbospondylus piscosus?}~\cauthyr{Leidy1868PANSP}

\emph{Dearcmhara shawcrossi}~\cauthyr{Brusatte2015SJG}

\emph{Eurhinosaurus longirostris}~\pauthyr{Mantell1851}

\emph{Excalibosaurus costini}~\cauthyr{McGowan1986N}

\emph{Gengasaurus nicosiai}~\cauthyr{Paparella2016GM}

\emph{Grippia longirostris}~\cauthyr{Wiman1929BGIU}

\emph{Guizhouichthyosaurus tangae}~\cauthyr{Yin2000GG}

\emph{Guizhouichthyosaurus wolonggangense}~\pauthyr{Chen2007GC}

\emph{Gulosaurus helmi}~\cauthyr{Cuthbertson2013JVP}

\emph{Hauffiopteryx typicus}~\cauthyr{Maisch2008P}

\emph{Himalayasaurus tibetensis}~\cauthyr{Young1972MNIGP}

\emph{Hudsonelpidia brevirostris}~\cauthyr{McGowan1995CJES}

\emph{Ichthyosaurus acutirostris}~\cauthyr{Owen1840RBAAS}

\emph{Ichthyosaurus anningae}~\cauthyr{Lomax2015JVP}

\emph{Ichthyosaurus breviceps}~\cauthyr{Owen1881MPS}

\emph{Ichthyosaurus communis}~\cauthyr{Conybeare1822TGSL}

\emph{Ichthyosaurus conybeari}~\cauthyr{Lydekker1888GM}

\emph{Ichthyosaurus larkini}~\cauthyr{Lomax2017PP}

\emph{Ichthyosaurus somersetensis}~\cauthyr{Lomax2017PP}

\emph{Isfjordosaurus minor}~\pauthyr{Wiman1910BGIUa}

\emph{Leninia stellans}~\cauthyr{Fischer2014GM}

\emph{Leptonectes moorei}~\cauthyr{McGowan1999P}

\emph{Leptonectes solei}~\pauthyr{McGowan1993CJES}

\emph{Leptonectes tenuirostris}~\pauthyr{Conybeare1822TGSL}

\emph{Macgowania janiceps}~\pauthyr{McGowan1996CJES}

\emph{Maiaspondylus lindoei}~\cauthyr{Maxwell2006P}

\emph{Malawania anachronus}~\cauthyr{Fischer2013BL}

\emph{Mikadocephalus gracilirostris}~\cauthyr{Maisch1997PZ}

\emph{Mixosaurus cornalianus}~\pauthyr{Bassani1886ASISN}

\emph{Mixosaurus kuhnschneyderi}~\pauthyr{Brinkmann1998NJGPA}

\emph{Mixosaurus xindianensis}~\cauthyr{Chen2010APS}

\emph{Mollesaurus pariallus}~\cauthyr{Fernandez1999JP}

\emph{Muiscasaurus catheti}~\cauthyr{Maxwell2016PP}

\begin{synonymy}
* & \cyear{Maxwell2016PP} & \emph{Muiscasaurus catheti} \cauth{Maxwell2016PP}& \crefauth{Maxwell2016PP}, p~61 [Arcillolitas abigarradas Member, Paja Formation (Barremian–Aptian, Lower Cretaceous, Cretaceous); Vereda Llanitos, Sachica, Boyaca, Colombia (\textallsc{UTM WGS84 18N 662860 616018 = 05° 34.278′ N 73° 31.781′ W}).]  \\
\end{synonymy}

\emph{Nannopterygius enthekiodon}~\pauthyr{Hulke1871QJGS}

\emph{Nannopterygius saveljeviensis}~\pauthyr{Arkhangelsky1997PZ}

\begin{synonymy}
* & \cyear{Arkhangelsky1997PZ} & \emph{Paraophthalmosaurus savejeviensis} \cauth{Arkhangelsky1997PZ}& \crefauth{Arkhangelsky1997PZ}, p~88  \\
* & \cyear{Efimov1999PZ} & \emph{Yasykovia kabanovi} \cauth{Efimov1999PZ}& \crefauth{Efimov1999PZ}, p~98  \\
 & \cyear{Zverkov2020ZJLS} & \emph{Nannopterygius savejeviensis} \pauth{Arkhangelsky1997PZ}& \crefauth{Zverkov2020ZJLS}, p~246 [Volgian, Upper Jurassic, Jurassic.]  \\
\end{synonymy}

\emph{Nannopterygius yasykovi}~\pauthyr{Efimov1999PZ}

\begin{synonymy}
* & \cyear{Efimov1999PZ} & \emph{Yasykovia sumini} \cauth{Efimov1999PZ}& \crefauth{Efimov1999PZ}, p~98  \\
\end{synonymy}

\emph{Ophthalmosaurus icenicus}~\cauthyr{Seeley1874QJGSa}

\begin{synonymy}
v * & \cyear{Seeley1874QJGSa} & \emph{Ophthalmosaurus icenicus} \cauth{Seeley1874QJGSa}& \crefauth{Seeley1874QJGSa}, p~707 [Peterborough Member, Oxford Clay Formation (Callovian, Middle Jurassic, Jurassic); Peterborough, Cambridgeshire.]  \\
\end{synonymy}

\emph{Ophthalmosaurus natans}~\pauthyr{Marsh1879AMNH}

\emph{Parvinatator wapitiensis}~\cauthyr{Nicholls1995Vfateosc}

\emph{Pervushovisaurus bannovkensis}~\cauthyr{Arkhangelsky1998PJ}

\emph{Pervushovisaurus campylodon}~\pauthyr{Carter1846RBAAS}

\begin{synonymy}
* & \cyear{Carter1846RBAAS} & \emph{Ichthyosaurus campylodon} \cauth{Carter1846RBAAS}& \crefauth{Carter1846RBAAS}, p~60 [Upper Greensand Formation (Albian–Cenomanian, Lower–Upper Cretaceous, Cretaceous); Cambridge,  Cambridgeshire, U.K..]  \\
 & \cyear{vonHuene1922} & \emph{Myopterygius campylodon} \pauth{Carter1846RBAAS}& \crefauth{vonHuene1922}, p~98  \\
 & \cyear{McGowan1972CGUW} & \emph{Platypterygius campylodon} \pauth{Carter1846RBAAS}& \crefauth{McGowan1972CGUW}, p~17  \\
 & \cyear{Fischer2016P} & \emph{Pervushovisaurus campylodon} \pauth{Carter1846RBAAS}& \crefauth{Fischer2016P}, p~8  \\
\end{synonymy}

\emph{Pessopteryx nisseri}~\cauthyr{Wiman1910BGIUa}

\emph{Phalarodon callawayi}~\cauthyr{Schmitz2004PAP}

\emph{Phalarodon fraasi}~\cauthyr{Merriam1910UCBDG}

\emph{Phalarodon major}~\cauthyr{vonHuene1916P}

\emph{Phantomosaurus neubigi}~\pauthyr{Sander1997AMR}

\emph{Platypterygius  hercynicus}~\cauthyr{Kuhn1946BNGBb}

\emph{Platypterygius americanus}~\cauthyr{Nace1939AJS}

\begin{synonymy}
* & \cyear{Nace1939AJS} & \emph{Myopterygius americanus} \cauth{Nace1939AJS}& \crefauth{Nace1939AJS}, p~674 [Mowry Shale Member, Graneros Formation; Crook County Wyoming, U.S.A..]  \\
 & \cyear{Romer1968CGUW} & \emph{Myopterygius americanus} \cauth{Nace1939AJS}& \crefauth{Romer1968CGUW}, p~27 [Mowry Shale Member, Graneros Formation; Osage, Wyoming, U.S.A..]  \\
 & \cyear{McGowan1972CGUW} & \emph{Platypterygius americanus} \pauth{Nace1939AJS}& \crefauth{McGowan1972CGUW}, p~17  \\
\end{synonymy}

\emph{Platypterygius australis}~\pauthyr{MCoy1867AMNH}

\emph{Platypterygius hauthali}~\pauthyr{vonHuene1927ZFMGPB}

\emph{Platypterygius ochevi}~\cauthyr{Arkhangelsky2008PJ}

\emph{Platypterygius platydactylus}~\pauthyr{Broili1907P}

\emph{Platypterygius sachicarum}~\cauthyr{Paramo1997RI}

\emph{Protoichthyosaurus applebyi}~\cauthyr{Lomax2017JVP}

\emph{Protoichthyosaurus prosaxalis}~\cauthyr{Appleby1979P}

\emph{Qianichthyosaurus xingyiensis}~\cauthyr{Yang2013ASNUP}

\emph{Qianichthyosaurus zhoui}~\cauthyr{Li1999CSB}

\emph{Quasianosteosaurus vikinghoegdai}~\cauthyr{Maisch2003NJGPAa}

\emph{Sclerocormus parviceps}~\cauthyr{Jiang2016SR}

\emph{Shastasaurus alexandrae}~\cauthyr{Merriam1902UCBDG}

\emph{Shastasaurus liangae}~\pauthyr{Yin2000GG}

\emph{Shastasaurus pacificus}~\cauthyr{Merriam1895AJS}

\emph{Shastasaurus sikkaniensis}~\pauthyr{Nicholls2004JVP}

\emph{Shonisaurus popularis}~\cauthyr{Camp1976SOeAW}

\emph{Simbirskiasaurus birjukovi}~\cauthyr{Otschev1985PZ}

\emph{Sisteronia seeleyi}~\cauthyr{Fischer2014PO}

\emph{Stenopterygius aaleniensis}~\cauthyr{Maxwell2012PO}

\emph{Stenopterygius quadriscissus}~\pauthyr{Quenstedt1858}

\emph{Stenopterygius triscissus}~\pauthyr{Quenstedt1858}

\emph{Stenopterygius uniter}~\cauthyr{vonHuene1931ASNG}

\emph{Suevoleviathan disinteger}~\pauthyr{vonHuene1926NJFMGPBBB}

\emph{Suevoleviathan integer}~\pauthyr{Bronn1844NJMGGP}

\emph{Sveltonectes insolitus}~\cauthyr{Fischer2011JVP}

\emph{Temnodontosaurus azerguensis}~\cauthyr{Martin2012P}

\emph{Temnodontosaurus crassimanus}~\pauthyr{Blake1876}

\emph{Temnodontosaurus eurycephalus}~\cauthyr{McGowan1974LSCROM}

\emph{Temnodontosaurus nuertingensis}~\pauthyr{vonHuene1931NJFMGPBB}

\emph{Temnodontosaurus platyodon}~\pauthyr{Conybeare1822TGSL}

\emph{Temnodontosaurus trigonus}~\pauthyr{Theodori1843GAKAWMC}

\emph{Thaisaurus chonglakmanii}~\cauthyr{Mazin1991CRASS2MPCSUST}

\emph{Thalattoarchon saurophagis}~\cauthyr{Frobisch2013PNAS}

\emph{Tholodus schmidi}~\cauthyr{Meyer1849P}

\emph{Toretocnemus californicus}~\cauthyr{Merriam1903UCBDG}

\emph{Toretocnemus zitteli}~\pauthyr{Merriam1903UCBDG}

\emph{Undorosaurus gorodischensis}~\cauthyr{Efimov1999PZa}

\begin{synonymy}
* & \cyear{Efimov1999PZa} & \emph{Undorosaurus gorodischensis} \cauth{Efimov1999PZa}& \crefauth{Efimov1999PZa}, p~52 [\emph{Epivirgatites nikitini} Ammonite Biozone (Volgian, Upper Jurassic, Jurassic); Undory, Volga Oblast, Russia.]  \\
 & \cyear{Druckenmiller2012NJG} & \emph{Cryopterygius kristiansenae} \cauth{Druckenmiller2012NJG}& \crefauth{Druckenmiller2012NJG}, p~313 [Slottsmøya Member, Agardfjellet Formation (middle Volgian, Upper Jurassic, Jurassic); Janusfjellet, Spitsbergen, Svalbard (\textallsc{UTM WGS84 33N 0518842 8696067}).] \emph{fide} \textcite{Zverkov2019JSP} \\
 & \cyear{Zverkov2019JSP} & \emph{Undorosaurus gorodischensis} \cauth{Efimov1999PZa}& \crefauth{Zverkov2019JSP}, p~1189  \\
\end{synonymy}

\emph{Undorosaurus kielanae}~\pauthyr{Tyborowski2016APP}

\begin{synonymy}
* & \cyear{Tyborowski2016APP} & \emph{Cryopterygius kielanae} \cauth{Tyborowski2016APP}& \crefauth{Tyborowski2016APP}, p~793 [Sławno Limestone Member, Kcynia Formation (uppermost Lower Tithonian = Middle Volgian, Upper Jurassic, Jurassic); Owadów-Brzezinki Quarry, Sławno (\textallsc{UTM WGS84 34U 439831 5692022 = 51.3762583 N 20.1355167 E}).] A test comment. \\
? & \emyear{Zverkov2019JSP} & \emph{Undorosaurus kielanae} \pauth{Tyborowski2016APP}& \crefauth{Zverkov2019JSP}, p~1187  \\
\end{synonymy}

\emph{Undorosaurus nessovi}~\cauthyr{Efimov1999PZa}

\emph{Undorosaurus trautscholdi}~\cauthyr{Arkhangelsky2014PZIR}

\emph{Utatsusaurus hataii}~\cauthyr{Shikama1978SRTUGa}

\emph{Wahlisaurus massarae}~\cauthyr{Lomax2016JSP}

\begin{synonymy}
* & \cyear{Lomax2016JSP} & \emph{Wahlisaurus massarae} \cauth{Lomax2016JSP}& \crefauth{Lomax2016JSP}, p~388 [Barnstone Member?, Scunthorpe Mudstone Formation?, Pre-\emph{planorbis} or \emph{Psiloceras planorbis} beds (lowermost Hettangian, Lower Jurassic, Jurassic); Normanton Hills near Normanton on Soar, Nottinghamshire, U.K..]  \\
\end{synonymy}

\emph{Wimanius odontopalatus}~\cauthyr{Maisch1998NJGPM}

\emph{Xinminosaurus catactes}~\cauthyr{Jiang2008PNS}




\mywidebib
% \printbibliography

\end{document}
